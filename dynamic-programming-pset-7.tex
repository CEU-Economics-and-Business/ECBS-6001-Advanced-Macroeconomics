\documentclass[12pt]{article}

% Packages for formatting
\usepackage[T1]{fontenc}
\usepackage{times}
\usepackage[utf8]{inputenc}
\usepackage{setspace}
\usepackage[margin=1in]{geometry}
\usepackage{amsmath, amssymb, amsfonts}
\usepackage{mathptmx}
\usepackage{microtype}
\usepackage{parskip}
\usepackage{listings}

\onehalfspacing

\title{\Large CEU -- Advanced Macroeconomics\\[0.5em]Problem Set 7}
\author{}
\date{}

\begin{document}
\maketitle

\noindent\textbf{Value Function Iteration in Julia} (20 points)

Consider a simple consumption-savings problem where an agent maximizes:
\[
\sup_{\{c_t\}_{t=0}^{\infty}} \sum_{t=0}^{\infty} \beta^t u(c_t)
\]
subject to:
\[
k_{t+1} = Rk_t - c_t
\]
\[
k_t \geq 0, c_t \geq 0 \text{ for all } t
\]

where $R = 1.05$ is the gross interest rate, $\beta = 0.95$ is the discount factor, and the utility function is:
\[
u(c) = \frac{c^{1-\theta} - 1}{1-\theta}
\]
with $\theta = 2$.

a. (8 points) Implement value function iteration in Julia to solve this problem:
   \begin{itemize}
   \item Create a \texttt{BellmanProblem} struct containing the problem parameters
   \item Define the utility function and constraint correspondence
   \item Write the Bellman operator function
   \item Implement the iterative solution method with a convergence criterion
   \end{itemize}

b. (8 points) Plot the following using any Julia plotting package:
   \begin{itemize}
   \item The converged value function
   \item The policy function for consumption
   \item The policy function for next period's capital
   \end{itemize}

c. (4 points) Extra points will be awarded for:
   \begin{itemize}
   \item Using multiple dispatch appropriately
   \item Writing clear docstrings for functions
   \item Using type annotations for variables
   \item Making the code modular and reusable
   \end{itemize}

Implementation notes:
\begin{itemize}
\item Use a grid of 100 points for capital, ranging from 0 to 20
\item Set convergence tolerance to $10^{-6}$
\item Initialize the value function to zero
\item Submit your code as a Julia script or Pluto notebook
\end{itemize}

Hints:
\begin{itemize}
\item Review the example code showing how to structure the Bellman iteration
\item The Plots.jl package is recommended for visualization
\item Test your code with a simpler utility function first
\end{itemize}

\end{document}
