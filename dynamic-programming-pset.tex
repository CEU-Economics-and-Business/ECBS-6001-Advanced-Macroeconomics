\documentclass[12pt]{article}

% Packages for formatting
\usepackage[T1]{fontenc}
\usepackage{times}  % Times New Roman font
\usepackage[utf8]{inputenc}
\usepackage{setspace}  % For line spacing
\usepackage[margin=1in]{geometry}  % Smaller margins
\usepackage{amsmath, amssymb, amsfonts}  % Math packages
\usepackage{mathptmx}  % Times New Roman for math mode
\usepackage{microtype}  % Better typography
\usepackage{parskip}  % Better paragraph spacing

% Set line spacing to 1.5
\onehalfspacing

% Custom theorem environments
\usepackage{amsthm}
\newtheorem{theorem}{Theorem}[section]
\newtheorem{lemma}[theorem]{Lemma}
\theoremstyle{definition}
\newtheorem{definition}[theorem]{Definition}

% Title formatting
\usepackage{titling}
\setlength{\droptitle}{-4em}
\title{CEU -- Advanced Macroeconomics\\Problem Set 6}
\author{}
\date{}

\begin{document}
\maketitle

\noindent\textbf{Question 1 -- Dynamic Programming Fundamentals}

Consider a simple infinite-horizon consumption problem where the agent maximizes:

\[
\sup_{\{c_t\}_{t=0}^{\infty}} \sum_{t=0}^{\infty} \beta^t \ln(c_t)
\]

subject to:
\[
k_{t+1} = Rk_t - c_t
\]
\[
k_t \geq 0, c_t \geq 0 \text{ for all } t
\]
\[
k_0 \text{ given}
\]

where $R > 1$ is the gross interest rate.

a. Write down the sequence problem formulation for this optimization problem.

b. Write down the corresponding Bellman equation.

c. Guess that the value function takes the form $v(k) = A + B\ln(k)$ where $A$ and $B$ are constants. Using this guess:
   i. Write down the first-order condition
   ii. Write down the envelope condition
   iii. Solve for the constants $A$ and $B$
   iv. Derive the optimal policy function for consumption

d. Verify that your solution satisfies the transversality condition.

\vspace{0.5cm}
\noindent\textbf{Question 2 -- Properties of the Bellman Operator}

Consider the Bellman operator $B$ defined as:

\[
(Bw)(x) = \sup_{x_{+1} \in \Gamma(x)} \{F(x,x_{+1}) + \beta w(x_{+1})\}
\]

a. Let $w_1$ and $w_2$ be two different functions. Show that if $w_1(x) \leq w_2(x)$ for all $x$, then $(Bw_1)(x) \leq (Bw_2)(x)$ for all $x$ (monotonicity).

b. Let $w$ be a function and $a$ be a constant. Show that:
\[
B(w + a)(x) = (Bw)(x) + \beta a
\]
(discounting)

c. Explain why these two properties (monotonicity and discounting) are important for proving that the Bellman operator is a contraction mapping.

\vspace{0.5cm}
\noindent\textbf{Question 3 -- Analytical Value Function Iteration}

Consider a simplified version of the growth model where:
\[
F(k,k_{+1}) = \ln(k - k_{+1})
\]
\[
\Gamma(k) = \{k_{+1}: 0 \leq k_{+1} \leq k\}
\]

a. Start with the initial guess $v_0(k) = 0$ for all $k$. Calculate:
   i. $(Bv_0)(k)$
   ii. $(B^2v_0)(k)$
   iii. $(B^3v_0)(k)$

b. Based on your calculations in part (a), conjecture a pattern for $(B^nv_0)(k)$ as $n \to \infty$.

c. Verify that your conjectured limit function satisfies the Bellman equation.

\end{document}
